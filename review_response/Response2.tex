%%//////////////////////////////////////////////////////////////
%%
%% P R E A M B L E
%%
%%//////////////////////////////////////////////////////////////
\documentclass{article}

%%==============================================================
%% Packages=
%%==============================================================
\usepackage[english]{babel}
\usepackage{booktabs}
\usepackage{scicite}
%\usepackage{citesort}
\usepackage[usenames,dvipsnames]{color}
\usepackage{graphics,amsfonts,amssymb,amsmath,latexsym}
\usepackage{graphicx,subfig}
\usepackage[ansinew]{inputenc}
%\usepackage{natbib}
\usepackage{multirow}
\usepackage{rotating}
\usepackage{setspace}
\usepackage{subfig}
\usepackage{textcomp}
\usepackage{hyperref}
\usepackage{lscape}
\usepackage{authblk}
\usepackage{xr}
\usepackage{enumerate}
\usepackage{bm}
\usepackage{epstopdf}
\usepackage{float}
\setlength\parindent{0pt}
\usepackage{xr}
\usepackage{csquotes}
\usepackage{enumitem}
\usepackage[dvipsnames]{xcolor}



\usepackage[usenames, dvipsnames]{color}
\usepackage[dvipsnames]{xcolor}

% Color Edits
\newcommand{\add}[1]{\noindent \color{blue} #1 \normalcolor}
% \newcommand{\del}[1]{\noindent \color{red} \st{#1}\normalcolor}
% \newcommand{\AM}[1]{\noindent \color{magenta} (AM: #1)\normalcolor}
% \newcommand{\AZ}[1]{\noindent \color{purple} (AZ: #1)\normalcolor}
% \newcommand{\PD}[1]{\noindent \color{cyan} (PD: #1)\normalcolor}


%\usepackage{scicite}
\usepackage{natbib}
\bibliographystyle{unsrt}


%%==============================================================
%% New Commands
%%==============================================================
\newcommand{\Hline}{\rule{\linewidth}{.1mm}}
%\newcommand{\RedColor}[1]{\noindent \color{red}  #1 \normalcolor}
\newcommand{\Question}[1]{\noindent \color{black}\emph{#1}\normalcolor}
\newcommand{\Answer}[1]{\noindent {\color{blue}{ #1}}\normalcolor}

\newcommand{\AnswerQ}[1]{\noindent {\footnotesize {\color{blue}{ #1}}}\normalcolor}

\renewcommand{\thepage}{R\arabic{page}}
\renewcommand{\thesection}{R\arabic{section}}
\renewcommand{\theequation}{R\arabic{equation}}
\renewcommand{\thefigure}{R\arabic{figure}}
\renewcommand{\thetable}{R\arabic{table}}
\renewcommand{\thetable}{R\arabic{table}}

\def\th{\theta}

\newcommand{\bfphi}{\boldsymbol{\phi}}%

\newcommand{\Hb}{\mathbf{\bar{H}}}

\newcommand{\revised}[1]{\noindent\color{Blue}#1\normalcolor}

\newcommand{\PD}[1]{\noindent \color{red} (PD: #1)\normalcolor}
\newcommand{\EM}[1]{\noindent \color{magenta} (EM: #1)\normalcolor}
\newcommand{\AZ}[1]{\noindent \color{purple}(AZ: #1)\normalcolor}

\def\p{\textit{p.~}}
\def\l{\textit{l.~}}
%%==============================================================
%% Setttings
%%==============================================================
\topmargin -0.5in


\setlength{\textwidth}{6in} \setlength{\textheight}{8.25in}
\setlength{\oddsidemargin}{0mm}
\setlength{\textwidth}{6.in}
\setlength{\textheight}{8.5in}
\setlength{\evensidemargin}{0.25in}
\setlength{\oddsidemargin}{0.25in}
\setlength{\oddsidemargin}{0mm}

%%//////////////////////////////////////////////////////////////
%%
%% T I T L E    &    A U T H O R S
%%
%%//////////////////////////////////////////////////////////////
%%%%%%%%%%%%%%%%%%%%%%%%%%%%
%%%%%%%%%%%%%%%%%%%%%%%%%%%%
\title{
Response to the Reviewers' comments for manuscript \\
(Tracking \#: LF18481): \\
``Temporal evolution of flow in pore-networks: From homogenization to instability"
}

\author{Ahmad Zareei, Deng Pan, and  Ariel Amir
}
\date{}

%%//////////////////////////////////////////////////////////////
%%
%% FIRST SENTENSE
%%
%%//////////////////////////////////////////////////////////////
\begin{document}
\maketitle

\noindent In the following, we address each of the Referees' comments/suggestions (\emph{in italic}). All page/paragraph/line (\textit{p.},\ \textit{par.},\ \textit{l.},\ etc.) numbers refer to the revised paper. %In addition, while using the original numbering of figures/equations/references in the manuscript (\emph{e.g.}, Fig.~1, Fig.~2,...., Eq.~(1), Eq.~(2),..., [1], [2], ...),  we add prefix ``\emph{R}'' for figures/equations/references presented in this response to the Referees' comments (\emph{e.g.}, Fig.~R1, Fig.~R2, .... Eq.~(R1), Eq.~(R2), ..., Ref.~[R1], Ref.~[R2],...).
% \centerline{(Tracking \#: LB15473 Deng)} 

\vspace{5 mm}
%%%%%%%%%%%%%%%%%%%%%%%%%%%%%%%%%%%%%%%%%%%%
% Ref 1
%%%%%%%%%%%%%%%%%%%%%%%%%%%%%%%%%%%%%%%%%%%%
\noindent
\Hline \\
\textbf{Response to Reviewer B} \\
\Hline
\\

% \centerline{Negative numbers are counted from the bottom of the page.}
%({\color{orange}\mbox{------}})

\Question{I have read the three referee reports
and the authors' answers to all questions and criticisms. I think the
authors have tried to answer all the referee's points and most answers
are satisfactory to me.}

\vspace{0.3 cm}

\Answer{We appreciate the referee's thorough reviews.}
\vspace{0.3 cm}

\Question{
The only answer that I do not find completely satisfying is to this
point: ``...since the authors have access to experimental setups, it
would greatly improve this work if they could provide experimental
proof of the phase transition in a real system." To what the authors
answered: `` With regards to the experimental tests, all the authors of
the current manuscript are theorists, and experimental studies of this
work is out of our scope." 2/3 of the authors of this manuscript are coauthors of Reference [31],
which presents experiments related to the results of the current
manuscript. Actually, one of the co-first authors of this manuscript
is also co-first author in [31]. I do not think that all publications
in PRL should include experimental results, but in this particular
case it is striking that the authors do not try to validate their
theory having (apparently) all the experimental tools at hand.
\newline
To summarize, the main and unanimous criticism from the first round of
review was if the paper was physically grounded. On the one hand, I
think the authors have greatly improved the manuscript including more
realistic models of physical erosion. However, on the other hand, I
think the authors have missed the opportunity of including at least
some preliminary experimental results that support their theory or at
least to discuss in their answer why this was not possible.
\newline
Finally, whereas I do not strongly support publication in PRL, I do
not oppose to it either, I leave the final decision to the editors.
\newline}

\Answer{We thank the referee for appreciating the more realistic physical
models utilized in the revised version and the extended theoretical
framework we used.
With regards to the question about the potential experimental tests of
the theory:
We completely agree with the referee that performing experiments to
test our theory is important. Unfortunately performing them is
currently unfeasible for us, since the Amir group is a theory group
and has no lab space.
We would like to make some clarifications with regards to Ref. [31].
In this work Dr. Shima Parsa, then a postdoc in the Weitz lab, had
empirical observations relating the mean velocity and permeability as
flow in a porous material is clogged with a polymer. Dr. Ahmad Zareei,
working under the guidance of Prof. Ariel Amir, set out to explain the
observed scaling using a combination of numerical simulations and
analytical work, which led to a joint publication (Ref. [31] mentioned
by the referee). Note that the experiments were performed by Dr. Parsa
over the course of several years while she was at Harvard. Dr. Parsa
has recently started as faculty at the Rochester Institute of
Technology, and therefore the experimental system associated with Ref.
[31] is no longer active at Harvard. We hope that as Referee D mentioned our theoretical study will trigger further experimental work in the future.}


\newpage
%%%%%%%%%%%%%%%%%%%%%%%%%%%%%%%%%%%%%%%%%%%%%%%%%%%%%%%%%%%%%%
%%%%%%%%%%%%%%%%%%%% Review # 2
%%%%%%%%%%%%%%%%%%%%%%%%%%%%%%%%%%%%%%%%%%%%%%%%%%%%%%%%%%%

\vspace{10 mm}
\noindent
\Hline \\
\textbf{Response to Reviewer C} \\
\Hline
\\


\Question{The work is greatly improved after revision, and many of the issues
raised by the reviewers have been effectively addressed. In particular
the consideration of nonlinear erosion models and the derivation of a
more general local stability criterion address many of the concerns
about generality and applicability, expressed by all referees.
\newline
For me, the modified work now achieves the level of significance that
it can be published PRL and my suggestions for further changes are
relatively minor. However, there would be significant advantages in
publishing in PRE or PRFluids where the excellent Supplementary
Material could be included in the body of the article.\newline}


\Answer{We thank the reviewer for their thoughtful comments.}



\vspace{0.5cm}
\textbf{Comment C1}
\noindent \vspace{-0.2cm}\\ \Hline\\

\Question{Having said that, the work does have a significant weakness in that
the primary model studied is a power law and therefore scale-free. The
scale-free nature is the primary reason why the local dynamics set the
behavior of the entire network. The majority of interesting real-world
cases will have a scale associated with them (e.g. a non-trivial
dependence on flow-rate). In the more general case we will find that
Eq. (3) and the condition described just after it will hold for some
flow rates but not all. This dependence on flow rate (or applied
pressure drop) is not well explained in the manuscript. Also, the new investigation of a tanh profile shows that intermediate states are
possible - i.e. states that lie somewhere between homogenization and
instability. It would be good for the manuscript to acknowledge some
of these limitations of their result - which remains interesting.
\newline}

\Answer{We thank the reviewer for the valuable comment. Indeed for a more general erosion dynamics the condition in Eq.~(3) might not hold for all the local connections and as noted a transitional region is observed; in other words, the stability criterion is obeyed initially for some pipes (below a threshold shear stress/radius) but does not hold above it. Naively, one would expect flow in the stable pipes to be maintained over time, while channelization would occur for the others. However, during the process of channelization, flow is redistributed throughout the network and the flow distribution therefore changes -- implying that some pipes that were initially in the stable regime may become unstable at later times. This is indeed what we observe in our simulations, for non-scale-free erosion laws. Furthermore, it is precisely this dynamics which enables the formation of \emph{multiple} channels, in contrast to the single channels observed for the scale-free erosion laws.

We have now included the remarks from the supplementary material in the main text. With regards to addressing the limitations, we have now expanded our conclusion to address the limitations of our model. The relevant parts from the main text are included here for completeness.\newline}
% our model has limitations
% . We have now expanded our conclusion to include such limitations as mentioned by the reviewer. admit scale free after Eq. 3 If not parallel, g(r) can be more complex acknowledge inbetween cases in the main text ...

\Answer{From \p3 of the main text:}
\AnswerQ{
\begin{quote}
   ``Note that, the fact that local dynamics predict the fate of the overall network in the way described above is a result of the scale-free nature of the aforementioned power-law erosion model. Later, we will discuss richer phenomenology that may arise in the case of physical erosion laws possessing an intrinsic scale.''
\end{quote}
}
\Answer{From \p4 of the main text:}
\AnswerQ{
\begin{quote}
    ``It is to be noted that the scale-free nature of our model as discussed after Eq. (3) is the main reason for the local dynamics setting the behavior of the whole network. Physical systems often have a scale associated with them, and in such cases, Eq.~(3) and the condition described just after hold true for some flow rates but not all. Investigating nonlinear erosion dynamics, we found that transitional states are then possible.''
\end{quote}
\AnswerQ{
\begin{quote}
    ``Note that for general erosion dynamics, the condition in Eq. (3) might hold for some but not all of the pipes at a given time. Furthermore, during the process of erosion, flow is redistributed throughout the network and therefore the flow distribution changes. This dynamical process enables the formation of \emph{multiple} channels, in contrast to the single channels observed for the scale-free erosion laws (see SM \S7 and Fig. S10).''
\end{quote}
}
}




\vspace{0.5cm}
\textbf{Comment C2}
\noindent \vspace{-0.2cm}\\ \Hline\\


\Question{In the abstract `First' in line 2 is a bit confusing as it isn't
followed up with a `second' or similar.\newline}


\Answer{We would like to thank the reviewer for carefully reading our manuscript. We removed `First' in the abstract to avoid any confusion.}
%
\\





%
    
\vspace{0.5cm}
\textbf{Comment C3}
\noindent \vspace{-0.2cm}\\ \Hline\\

\Question{On page 2 column 1, in the shear model, it's unclear to me why
moving from $\dot{m}$ to $\dot{r}$ would introduce an $r$ factor ($-\kappa$
becomes $-\kappa r$). Also it would be nice to add a reference for the
shear stress being proportional to $q/r^3$ - this is a standard result
but it would improve readability for the PRL audience.
\newline}

\Answer{We would like to thank the reviewer for their thorough reading. In fact, moving from $\dot{m}$ to $\dot{r}$, a factor of $1/r$ is introduced, since $dm=2\pi \rho rl dr$ in a cylindrical pipe. There is a typo in this equation and the factor should read $\kappa/r$. We have corrected the equation and additionally included more information to make the derivation clearer. The reference for the shear stress proportionality with $q/r^3$ is now added to the main text. The relevant part of the manuscript is included below.}
%
\\

\Answer{From \p2 of the main text:}
\AnswerQ{
\begin{quote}
    ``In a class of erosion models previously studied [34-36] the eroded mass per unit area is linearly proportional to the excess shear, i.e.,  $\dot{m}  = \kappa  (\tau_w-\tau_c)$ where $\kappa$ is the linear proportionality constant. This erosion model effectively results in an increase in the radius of the edges in our network model where $\dot{r} \propto \kappa/r (|q|/r^3 - \tau_c)$ , where the $1/r$ factor comes from the eroded mass $dm =2\pi\rho r l dr$ for a cylindrical pipe and $|q|/r^3$ comes from the shear stress of laminar flow in pipes [37].''
\end{quote}
}




\vspace{0.5cm}
\textbf{Comment C4}
\noindent \vspace{-0.2cm}\\ \Hline\\

\Question{Last line of page 3 - the parameter $L$ looks to be used without
definition.}
\\

\Answer{We thank the reviewer for the comment. We have now included the definition in the main text.}%

\Answer{From \p3 of the main text:}
\AnswerQ{
\begin{quote}
    ``To understand the transition in network behavior during erosion, we focus on a simplified model with only two tubes of the same length $L$ in parallel or series with a general erosion dynamics as $dr_{i}/dt = f(q_{i},r_{i}), i=1,2$.''   
\end{quote}
}





\vspace{0.5cm}
\textbf{Comment C5}
\noindent \vspace{-0.2cm}\\ \Hline\\

\Question{It would be great to briefly explain the physical origin of the
$r_1/r_2$ factor in equation 3 as this is such a significant part of the
result. My interpretation is that it is because stability depends on
the fractional change in conductance associated with a change in
radius, and that the fractional change will be inversely proportional
to the radius.\newline}

\Answer{We thank the reviewer for their valuable suggestion. The physical origin of the scale-free criteria $r_1/r_2$ turns out to be a result of power-law dependence of conductance with respect to radius (i.e., $C\propto r^k$). Interestingly, this indicates that our homogenization condition can be generalized to non-laminar flow where conductance has again a power-law dependence on the radius [50]. The relevant part is attached here for completeness. \newline }
\\

\Answer{From \p3 of the main text:}
\AnswerQ{
\begin{quote}
    ``Note that the scale-free criteria here (i.e., the dependence on the dimensionless factor $r_1/r_2$) results from the power-law dependence of conductance on radius, i.e., $C\propto r^k (k > 0)$. As a result, in addition to laminar flow (where $C\propto r^4$), the above condition also holds true for nonlaminar flow examples such as molecular flow where $C\propto r^3$ [50].''
    % ``In fact, this condition will hold for nonlaminar flow as long as . The power law dependence of conductance with respect to radius leads to the scale-free criteria $r_1/r_2$. ''
\end{quote}
}

%%%%%%%%%%%%%%%%%%%%%%%%%%%%%%%%%%%%%%%%%%%%%%%%%%%%%%%
%%%%%%%%%%%%%%%%%%%% Review # 3 %%%%%%%%%%%%%%%%%%%%%%%
%%%%%%%%%%%%%%%%%%%%%%%%%%%%%%%%%%%%%%%%%%%%%%%%%%%%%%%

\clearpage
% \bigbreak

% \bibliographystyle{unsrt}
% \bibliography{ref}

\end{document}